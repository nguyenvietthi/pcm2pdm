\phantomsection\addcontentsline{toc}{section}{\numberline {}KẾT LUẬN}
\section*{KẾT LUẬN} \label{ketluan}
\phantomsection\addcontentsline{toc}{section}{\numberline {}Kết luận chung}
\subsection*{Kết luận chung}

Qua quá trình nghiên cứu và thiết kế, bộ chuyển đổi PDM sang PCM nhiều giai đoạn đã đáp ứng được các yêu cầu thiết kế đặt ra. Tuy nhiên không thể tránh khỏi như sai lệch so với hệ thống triển khai bằng phần mềm do vấn đề về tính toán fixed-point.

Việc triển khai trên FPGA hoạt động ổn định và đáp ứng được các chức năng, chứng tỏ thiết kế có thể giao tiếp với các lõi công nghệ khác trong một hệ thống lớn.

\phantomsection\addcontentsline{toc}{section}{\numberline {}Hướng phát triển}
\subsection*{Hướng phát triển}
Mục tiêu của đề tài là sẽ tiếp tục cải tiến kiến trúc để tăng độ suy hao của dải dừng, giúp bộ chuyển đổi có hệ số SNR cao hơn. Từ cơ sở đó, có thể thực hiện các bước tiếp theo của quá trình thiết kế ASIC nhằm đưa ý tưởng thành sản phẩm thương mại hóa hoặc có thể nhúng nó vào hệ thống "System On Chip" lớn hơn.


Một lần nữa em xin gửi lời cảm ơn chân thành tới PGS. TS NGUYỄN ĐỨC MINH đã hướng dẫn nhiệt tình. Bên cạnh đó, em cũng xin gửi lời cảm ơn sâu sắc đến công ty \textbf{Dolphin Technology VietNam Center} đã giúp đỡ về mặt thiết bị trong quá trình thực thi đồ án này.

\newpage