\phantomsection\addcontentsline{toc}{section}{\numberline {}CHƯƠNG 1.  CHƯƠNG MỞ ĐẦU}
\section*{CHƯƠNG 1.  CHƯƠNG MỞ ĐẦU}
Micro âm thanh kỹ thuật số là một trong những cảm biến phổ biến nhất được sử dụng trong cuộc sống hàng ngày của con người. Trong số tất cả các loại đó, micro MEMS đang dần trở nên phổ biến do kích thước nhỏ hơn so với micro dòng điện dung truyền thống. Mật độ xung (PDM) là giao diện đầu ra thông dụng nhất của micro MEMS vì nó yêu cầu ít xử lý tín hiệu trên micro và giúp đồng bộ hóa micro dễ dàng hơn \cite{mems}. Tuy nhiên, tín hiệu PDM không hữu ích như tín hiệu mã xung (PCM) thường được sử dụng nhiều hơn để phân tích và tái tạo âm thanh đầu vào, vì vậy đầu ra từ micro kỹ thuật số PDM cần được chuyển đổi sang tín hiệu PCM trước khi được xử lý tiếp. Theo phương pháp truyền thống, việc chuyển đổi tín hiệu PDM sang PCM được thực hiện bằng phần mềm hoặc bộ xử lý tín hiệu kỹ thuật số (DSP).

Tuy nhiên, xu hướng những năm gần đây, FPGA được ưa chuộng hơn DSP trong nhiều nghiên cứu sử dụng mảng micro kỹ thuật số lớn. FPGA có nhiều ưu điểm hơn DSP khi xử lý dữ liệu từ một mảng micro lớn, cung cấp nhiều cổng IO hơn và cho phép mức độ song song phần cứng cao hơn. Và đặc biệt sau đó, có thể triển khai thiết kế thành "CHIP" (ASIC - một vi mạch IC được thiết kế dành cho một ứng dụng cụ thể) và thương mại hóa với giá thành rẻ hơn FPGA.

Nắm bắt được điều đó, trong đồ án này sẽ trình bày một "\textbf{hệ thống chuyển đổi PDM sang PCM nhiều giai đoạn}" sử dụng các bộ lọc tối ưu tài nguyên cho thiết kế phần cứng như bộ Cascaded Integrator-Comb (CIC), Half-Band.

\noindent \textbf{Mục đích}

Đề tài có các mục đích chính như sau:
\begin{itemize}
    \item Hiểu rõ được tín hiệu đầu vào PDM được điều chế từ bộ Sigma-Delta. Từ đó ánh xạ với thiết bị micro PDM thực tế để đưa ra các tiêu chí và yêu cầu thiết kế của hệ thống.
    \item  Tổng hợp ASIC với các yêu cầu kỹ thuật nhất định (tần số tối đa, số lượng cell tối đa) 
    \item Đưa ra hướng tích hợp bộ chuyển đổi lên các thiết bị phần cứng, đồng thời chứng minh hiệu quả thực tế bằng cách triển khai đánh giá trên FPGA.
    \item Tạo tiền đề để phát triển dự án thành "CHIP" trong tương lai.
\end{itemize}

\noindent \textbf{Phương pháp}

Kết quả sẽ được so sánh và đối chiếu với lý thuyết  mô phỏng bằng ngôn ngữ \textbf{Python} và thực thi trên thiết bị thật (FPGA). Từ đó, có thể đánh giá hệ thống một cách chính xác nhất về hiệu suất, tính ổn định, độ tin cậy và các yếu tố khác liên quan đến hệ thống đó.

\noindent \textbf{Phạm vi nghiên cứu}

Đồ án tập trung vào giải pháp tối ưu và hoàn thiện bài toán chọn kiến trúc nhằm cải thiện tài nguyên cho phần cứng. Kết quả thu được là cơ sở cũng như đầu vào cho các bước tiếp theo của quá trình thiết kế ASIC, từ đó đưa ý tưởng thành sản phẩm thương mại hóa.

\noindent Phần còn lại của đồ án được trình bày theo thứ tự sau:
\begin{itemize}
    \item \hyperref[chuong2]{Chương 2}: trình bày về cở sở lý thuyết về chuyển đổi tương tự số, các kỹ thuật xử lý tín hiệu trên miền tần số, các bộ lọc tối ưu, \ldots
    \item \hyperref[chuong3]{Chương 3}: trình bày về cách thiết kế và kiểm thử bằng phần mềm bộ chuyển đổi nhiều giai đoạn với các yêu thiết kế từ sản phẩm thực tế (MEMS).
    \item \hyperref[chuong4]{Chương 4}: thiết kế bộ chuyển đổi dưới mức register transfer logic (RTL) và kiểm thử tính đúng đắn.
    \item \hyperref[chuong5]{Chương 5}: tổng hợp và triển khai bộ chuyển đổi lên FPGA.
    \item \hyperref[ketluan]{Chương kết luận}: kết luận chung của đồ án
\end{itemize}
\newpage