\section*{LỜI NÓI ĐẦU}
\thispagestyle{empty}
MEMS (Microelectromechanical Systems) microphone là một loại microphone được sử dụng phổ biến trong các ứng dụng di động, thiết bị IoT (Internet of Things) và các hệ thống điều khiển giọng nói. Với kích thước nhỏ gọn và tiêu thụ điện năng thấp, MEMS microphone đã thay thế các loại microphone truyền thống và trở thành một công nghệ quan trọng trong ngành công nghiệp âm thanh.

MEMS microphone hoạt động bằng cách chuyển đổi sóng âm thành tín hiệu điện thông qua các cảm biến nhỏ được tích hợp trên một vi mạch nhỏ. Sự khác biệt của nó so với các loại microphone khác là chúng sử dụng kỹ thuật Pulse Density Modulation (PDM) để chuyển đổi tín hiệu tương tự sang số. PDM cho phép các tín hiệu âm thanh được biểu diễn bằng mật độ xung, giúp giảm thiểu tối đa nhiễu và giảm độ trễ trong việc xử lý tín hiệu.

Ngoài ra, MEMS microphone còn có nhiều ưu điểm khác như độ nhạy cao, khả năng tái tạo âm thanh chính xác và khả năng giảm tiếng ồn tốt. Với những ưu điểm này, MEMS microphone trở thành một công nghệ quan trọng trong việc cải thiện chất lượng âm thanh trong các sản phẩm điện tử tiêu dùng, từ điện thoại thông minh, máy tính bảng, đồng hồ thông minh cho đến loa thông minh và các thiết bị đo lường.

Với sự phát triển của công nghệ, MEMS microphone dự kiến sẽ tiếp tục cải tiến và phát triển để đáp ứng nhu cầu ngày càng cao về chất lượng âm thanh và tiêu thụ năng lượng thấp trong các ứng dụng di động và IoT (Internet of Things).

Nhận thấy được tiềm năng to lớn đó, cũng như qua những trải nghiệm thực tế và kiến thức được tiếp thu trong suốt quá trình học tập tại Đại Học Bách Khoa Hà Nội, em đã lựa chọn, tìm hiểu, nghiên cứu về bộ chuyển đổi PDM (Pulse density modulation) sang PCM (Pulse-code modulation) bằng thiết kế số và đưa vào đồ án tốt nghiệp.

Em xin chân thành cảm ơn  \textbf{PGS. TS. NGUYỄN ĐỨC MINH} - giảng viên Trường Điện –
Điện tử, Đại học Bách Khoa Hà Nội đã tận tình dẫn dắt, đưa ra 
các nhận xét trong quá trình thực thi đồ án. Đồng thời, em cũng cảm ơn anh \textbf{Nguyễn Hải Anh}, anh \textbf{Tạ Xuân Tùng} và công ty \textbf{Dolphin Technology Viet Nam Center} đã nhiệt tình hướng dẫn, tạo điều kiện và hỗ trợ các công cụ, thiết bị thiết kế để em có thể hoàn thiện dự án này. Do kiến thức còn hạn chế nên đề tài không tránh khỏi thiếu sót và sơ suất. Em rất mong nhận được các ý kiến đóng góp, nhận xét của các thầy cô để đề tài được hoàn thiện và chỉn chu hơn.

Em xin chân thành cảm ơn !

\hspace{7cm}Hà Nội, ngày 03 tháng 03 năm 2023

\newpage