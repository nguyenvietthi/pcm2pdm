% ĐÁNH GIÁ CỦA CÁN BỘ PHẢN BIỆN
{
\begin{center}\vspace{-15pt}
\fontsize{12pt}{0pt}\selectfont TRƯỜNG ĐẠI HỌC BÁCH KHOA HÀ NỘI \\
\vspace{0.2cm}
\textbf{\underline{\fontsize{12pt}{0pt}\selectfont TRƯỜNG ĐIỆN - ĐIỆN TỬ}}
\vspace{1.0cm}
\end{center}

\section*{\fontsize{14pt}{0pt}\selectfontĐÁNH GIÁ QUYỂN ĐỒ ÁN TỐT NGHIỆP\\\fontsize{12pt}{0pt}\selectfont \vspace{4pt}\textbf{(DÀNH CHO CÁN BỘ PHẢN BIỆN)}}
\thispagestyle{empty}
% \vspace{-10pt}

\noindent Tên đề tài: THIẾT KẾ, TỔNG HỢP VÀ TRIỂN KHAI BỘ CHUYỂN ĐỔI PDM SANG PCM \\
\vspace{0.2cm}
% \noindent\vbox spread 6pt {}\null\xleaders \hbox to 2mm {\hss . \hss}\hfill \null \\
\vspace{0.2cm}
\noindent Họ và tên SV: NGUYỄN VIỆT THI \hspace{3cm} MSSV: 20182798 \\
\noindent Cán bộ phản biện:\dotfill \\

\begin{table}[H]
    \centering
    \begin{tabular}{
    | P{0.05\linewidth} 
    | P{0.13\linewidth} 
    | P{0.63\linewidth} 
    | P{0.08\linewidth} |
    }
    \hline
        \textbf{STT}& \textbf{Tiêu chí \qquad \qquad \textnormal{(Điểm tối đa)}} & \textbf{Hướng dẫn đánh giá tiêu chí} & \textbf{Điểm tiêu chí} \\\hline

        \multirow{2}{*}{1} & \multirow{2}{*}{\parbox{2.15cm}{\vspace{0.2cm}\centering \textbf{Trình bày quyển ĐATN \qquad \quad (4 điểm)}}} & \raggedright Đồ án trình bày đúng mẫu quy định, bố cục các chương logic và hợp lý: Bảng biểu, hình ảnh rõ ràng, có tiêu đề, được đánh số thứ tự và được giải thích hay đề cập đến trong đồ án, có căn lề, dấu cách sau dấu chấm, dấu phẩy, có mở đầu chương và kết luận chương, có liệt kê tài liệu tham khảo và có trích dẫn, v.v. & \multirow{2}{*}{} \\ \cline{3-3}
         & & \raggedright Kỹ năng diễn đạt, phân tích, giải thích, lập luận: cấu trúc câu rõ ràng, văn phong khoa học, lập luận logic và có cơ sở, thuật ngữ chuyên ngành phù hợp, v.v. & \\\hline 

        \multirow{3}{*}{2} & \multirow{3}{*}{\parbox{2.15cm}{\vspace{0.1cm}\centering \textbf{Nội dung và kết quả đạt được  \qquad \quad (5.5 điểm)}}} & \raggedright Nêu rõ tính cấp thiết, ý nghĩa khoa học và thực tiễn của đề tài, các vấn đề và các giả thuyết, phạm vi ứng dụng của đề tài. Thực hiện đầy đủ quy trình nghiên cứu: Đặt vấn đề, mục tiêu đề ra, phương pháp nghiên cứu/ giải quyết vấn đề, kết quả đạt được, đánh giá và kết luận. & \multirow{3}{*}{} \\ \cline{3-3}
         & & \raggedright Nội dung và kết quả được trình bày một cách logic và hợp lý, được phân tích và đánh giá thỏa đáng. Biện luận phân tích kết quả mô phỏng/ phần mềm/ thực nghiệm, so sánh kết quả đạt được với kết quả trước đó có liên quan. & \\ \cline{3-3}
         & & \raggedright Chỉ rõ phù hợp giữa kết quả đạt được và mục tiêu ban đầu đề ra đồng thời cung cấp lập luận để đề xuất hướng giải quyết có thể thực hiện trong tương lai. Hàm lượng khoa học/ độ phức tạp cao, có tính mới/ tính sáng tạo trong nội dung và kết quả đồ án. & \\\hline
         
         3 & \multirow{2}{*}{\parbox{2.15cm}{\vspace{0.1cm}\centering \textbf{Điểm thành tích  \\ (1 điểm)}}} & \raggedright Có bài báo KH được đăng hoặc chấp nhận đăng/ đạt giải SV NCKH giải 3 cấp Trường trở lên/ Các giải thưởng khoa học trong nước, quốc tế từ giải 3 trở lên/ Có đăng ký bằng phát minh sáng chế. (1 điểm) & \\
  
         
    \end{tabular}
\end{table}
\begin{table}[H]
    \centering
    \begin{tabular}{
    | P{0.05\linewidth} 
    | P{0.13\linewidth} 
    | P{0.63\linewidth} 
    | P{0.08\linewidth} |
    }
         & & \raggedright Được báo cáo tại hội đồng cấp Trường trong hội nghị SV NCKH nhưng không đạt giải từ giải 3 trở lên/ Đạt giải khuyến khích trong cuộc thi khoa học trong nước, quốc tế/ Kết quả đồ án là sản phẩm ứng dụng có tính hoàn thiện cao, yêu cầu khối lượng thực hiện lớn. \textbf{(0,5 điểm)} & \\\hline 
         & & \raggedleft \textbf{Điểm tổng các tiêu chí:} & \\\hline
         & & \raggedleft \textbf{Điểm phản biện:} & \\\hline
    \end{tabular}
\end{table}

% \hspace{9cm} Ngày: ... / ... / 20...

\vspace{0.3cm}

\hspace{9.0cm}\textbf{Cán bộ phản biện}

\hspace{9cm}(Ký và ghi rõ họ tên)
\cleardoublepage
}